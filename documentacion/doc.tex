\documentclass[11pt]{article}

\usepackage[T1]{fontenc}
\usepackage[utf8]{inputenc}
\usepackage[spanish]{babel}
\usepackage[letterpaper,
            portrait,
            margin=2cm]{geometry}
\usepackage{graphicx}
\usepackage{textcomp}
\usepackage{enumerate}
\usepackage{enumitem}
\usepackage{alltt}
\usepackage{import}

\setlist[itemize]{noitemsep, topsep=0pt}

\newcommand{\bt}{\begin{alltt}}
\newcommand{\et}{\end{alltt}}
\newcommand{\comando}[2]{
    \textbf{#1}(#2)\\
}

\newenvironment{args}{
    \newline
    Argumentos:
    \begin{itemize}
}{
    \end{itemize}
    \bigskip
}

\setlength{\parindent}{0pt}

\title{Inteligencia Artificial 2018-1 \\ Proyecto2: Búsqueda\\
\vspace{2mm}
\small{IIMAS-PCIC}}
\author{Luis Alejandro Lara Patiño\\Roberto Monroy Argumedo\\
Alejandro Ehécatl Morales Huitrón}
\date{23 de noviembre de 2017}


\begin{document}

\maketitle

\tableofcontents

\section{Funcionamiento del proyecto}

Este proyecto contiene ..

\section{Módulos}

\subsection{Módulo de diagnóstico}

Para hacer un dignostico se necesita un conjunto de observaciones y una
locación actual. También una lista de objetos y posiciones restantes.
La forma en como se obtiene es realizando una búsqueda en la que en cada
paso agregamos una acción ´´mover'' o ´´colocar'' hasta que la lista de
objetos y posiciones restantes sea vacíá en cuyo caso agregamos la
locación inicial para así obtener (en una lista) el conjunto de acciones
de final a principio que realizó el almacenista.

La forma de nuestra función sucesor toma en cuenta la minima falla que pudo
cometer el almacenista, es decir suponemos que los productos están en su
lugar.

\subsection{Módulo de toma de desición}

\subsection{Módulo de planeación}


\subsection{Utilitarios}

\end{document}